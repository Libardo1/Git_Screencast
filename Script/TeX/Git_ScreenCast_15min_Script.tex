\documentclass{screenplay} % nice docs re this class here: <https://www.ctan.org/tex-archive/macros/latex/contrib/screenplay>
\usepackage{amsmath,authblk,hyperref}
\title{Version Control with the Git Command Line Application $\mu$-Tutorial}
\author[1,2,3]{Ben R. Fitzpatrick, PhD Candidate}
\affil[1]{\small Bayesian Research \& Applications Group, Mathematical Sciences School, Queensland University of Technology (QUT)}
\affil[2]{\small Cooperative Research Centre for Spatial Information, Victoria}
\affil[3]{\small Austrailan Research Council Centre of Excellence in Mathematical Statistical Frontier for Big Data, Big Models and New Insights (ACEMS)}
\begin{document}
\maketitle

\intslug{Computer Screen Capture} % what should I use as a slug line to introduce a screen capture?
Web browser displaying \url{www.git-scm.com}

\begin{dialogue}[warmly]{Narrator} Hello and welcome to this screencast micro-tutorial on the Git Version Control Software. \end{dialogue}

Switching tabs to \url{https://bragqut.wordpress.com/} and mousing over the Members $\rightarrow$ Ben Fitzpatrick Menu Entries

\begin{dialogue}{Narrator} My name is Ben Fitzpatrick and I'm PhD Student in QUT's Bayesian Research and Applications Group. \end{dialogue}

Switching tabs to \url{http://acems.org.au/}

\begin{dialogue}{Narrator} I am also aligned with the Australian Research Council Centre of Excellence for Mathematical and Statistical Frontiers in Big Data, Big Models and New Insights. \end{dialogue}

Switching tabs to \url{www.git-scm.com}

\begin{dialogue}{Narrator} Git is version control software.
\newline
\newline
If you have ever put numbers at the ends of file names to keep copies of different stages in your process of authoring those files you have used version control.
\newline
\newline
Furthermore, if you have have ever used comment boxes or tracked changes to collaboratively author a file then you have also used version control.
\newline
\newline
Version control software formalises these concepts into a system which can save you a lot of time by enforcing an organisational structure and providing functionality for automated operations within this structure.
\newline
\newline
Before I commence my demonstration of a simple workflow with the Git version control system it's worth mentioning that both Git and GitHub are exceptionally well documented online.
\end{dialogue}

\begin{dialogue}[navigating to the Git Book page]{Narrator} 
There is an entire book available on Git,
\end{dialogue}

\begin{dialogue}[switching tabs to \url{https://help.github.com/}]{Narrator} 
and the GitHub help page is also excellent.
\end{dialogue}

\begin{dialogue}[switching tabs to \url{https://github.com/brfitzpatrick/Intro_to_R/blob/master/VC_with_GitHub.pdf}]{Narrator} 
Furthermore, if you find this micro-tutorial helpful I have written a longer tutorial on version control with Git and GitHub the slides for which are freely available online.
\end{dialogue}

\begin{dialogue}[switching tabs to \url{https://github.com/brfitzpatrick/}]{Narrator} 
Today I'll demonstrate the simple workflow of creating a local clone of a remote repository, making changes to your local clone then pushing those changes to the remote to bring it up to date with your local version.
\end{dialogue}

Switch to terminal and demo git clone commit push

\begin{dialogue}[switching tabs to \url{https://github.com/brfitzpatrick/Intro_to_R/blob/master/VC_with_GitHub.pdf}]{Narrator} If you would like to follow along with the tutorial on your computer please ensure you have the Git command line application installed.  Installation instructions for MacOS, Microsoft Windows are various GNU plus Linux distributions are given at the beginning of the slides for my full Git tutorial.
\end{dialogue} 

\begin{dialogue}[switching tabs to \url{www.github.com}]{Narrator}
To follow along with this tutorial you will also need to have created an account on either GitHub or BitBucket.
\end{dialogue}

\begin{dialogue}[Signing into GitHub \& Creating a Repository]{Narrator} Open your web browser and navigate to the GitHub (or BitBucket) page and sign in.
\newline
\newline
Create a new repository.
\newline
\newline
Choose whether to make this repository Public or Private.
\newline
\newline
Initialize the repository with a Readme document to contain instructions for your users.
\newline
\newline
Choose a liscence to govern use and distribution of the materials contained in the repository.
You may find GitHub's summaries of the various liscences it includes by default useful here.
\newline
\newline
Click the green `Create' button to create your repository on the GitHub servers.
\end{dialogue}

\begin{dialogue}[Copy the HTTPS clone URL]{Narrator}
At the bottom right of the GitHub page for your new repository you will find the a button to copy the URLs to clone you repository to you computer's hard drive via HTTPS, SSH and Subversion.
\newline
\newline
Click the hyperlink titled `HTTPS' then click the `Copy to Clipboard' button'.
\end{dialogue}

\begin{dialogue}[Opening a terminal]{Narrator}
Open a command line interface.
\newline
\newline
On Windows open the `Start' Menu, choose `Program' then open the `PowerShell' program.
\newline
\newline
On MacOS open the `Finder' and search your `Applications' for the `Terminal' and open this.
\newline
\newline
GNU plus Linux users, this will vary a little from distribution to distribution but you should all be able to open the `Terminal' application from your applications list.
\end{dialogue}



\begin{dialogue} This concludes the demonstration.
\newline
\newline
Hopefully you now understand the basics of version control with the Git command line application and local and remote versions of a repository. % summarise key learning points
\end{dialogue}

\begin{dialogue}[joking \& switching to \url{http://git-scm.com/downloads/guis}]{Narrator}
If you are writing code and yet are averse to using a command line interface numerous Graphical User Interfaces to the Git version control system exist.
\newline
\newline
Of particular note are the clients from GitHub and Atlassian (the provider of BitBucket).
\end{dialogue}

\begin{dialogue}[sign off, switching back to \url{http://git-scm.com}]{Narrator}
Thanks for watching, my name is Ben Fitzpatrick and with Git `everything is local'.
\end{dialogue}



%\begin{dialogue}{Narrator}  \end{dialogue}

%\begin{dialogue}{Narrator}  \end{dialogue}

%\begin{dialogue}{Narrator}  \end{dialogue}

%\begin{dialogue}{Narrator}  \end{dialogue}

%\begin{dialogue}{Narrator}  \end{dialogue}

%\begin{dialogue}{Narrator}  \end{dialogue}

%\begin{dialogue}{Narrator}  \end{dialogue}

%\begin{dialogue}{Narrator}  \end{dialogue}

%\begin{dialogue}{Narrator}  \end{dialogue}

%\begin{dialogue}{Narrator}  \end{dialogue}

\fadeout
\theend

\end{document}